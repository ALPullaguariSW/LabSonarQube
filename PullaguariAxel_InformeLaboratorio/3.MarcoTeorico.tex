\chapter{Marco Teórico}

\section{Análisis Estático de Código (SAST)}
El Análisis Estático de Pruebas de Seguridad de Aplicaciones (SAST, Static Application Security Testing) es una metodología de prueba "White-Box" que analiza el código fuente, el código de bytes o los binarios de una aplicación en busca de condiciones de seguridad y calidad indicativas de vulnerabilidades. A diferencia del análisis dinámico (DAST), SAST se realiza sin ejecutar la aplicación, lo que permite identificar errores en fases tempranas del ciclo de desarrollo (Shift-Left Testing).

\section{SonarQube}
SonarQube es una plataforma de código abierto para la inspección continua de la calidad del código. Realiza revisiones automáticas para detectar errores, olores de código (code smells) y vulnerabilidades de seguridad en más de 20 lenguajes de programación.

\subsection{Arquitectura}
\begin{itemize}
    \item \textbf{SonarQube Server}: El núcleo que procesa los informes de análisis, gestiona la base de datos y sirve la interfaz web.
    \item \textbf{SonarScanner}: El agente que se ejecuta en la máquina del desarrollador o en el servidor de CI/CD. Analiza el código fuente línea por línea y envía el informe resultante al servidor.
    \item \textbf{Base de Datos}: Almacena las métricas, problemas y configuraciones (en este laboratorio, PostgreSQL).
\end{itemize}

\section{Métricas de Calidad de Software}
SonarQube basa su análisis en tres pilares fundamentales:

\subsection{Fiabilidad (Reliability)}
Mide la capacidad del software para realizar sus funciones requeridas bajo condiciones específicas. En SonarQube, esto se traduce en la detección de \textbf{Bugs}: errores de programación que pueden provocar fallos en tiempo de ejecución (ej. NullPointerExceptions, bucles infinitos).

\subsection{Seguridad (Security)}
Se divide en dos categorías:
\begin{itemize}
    \item \textbf{Vulnerabilidades}: Fallos confirmados que pueden ser explotados por atacantes (ej. Inyección SQL, XSS). Requieren corrección inmediata.
    \item \textbf{Security Hotspots}: Fragmentos de código sensibles a la seguridad que requieren revisión manual para confirmar si son seguros o no (ej. configuración de CORS, uso de cookies).
\end{itemize}

\subsection{Mantenibilidad (Maintainability)}
Relacionada con la facilidad con la que el código puede ser modificado. SonarQube utiliza el concepto de \textbf{Code Smells} (Olores de Código): patrones que no son necesariamente errores, pero indican debilidades de diseño o dificultades de mantenimiento (ej. código duplicado, funciones demasiado complejas, "magic numbers"). Se mide a través de la \textbf{Deuda Técnica} (el tiempo estimado para corregir estos problemas).

\section{Cobertura de Código (Code Coverage)}
Es una métrica que indica el porcentaje de código fuente que es ejecutado durante las pruebas automatizadas (Unit Testing).
\[ \text{Cobertura} = \frac{\text{Líneas ejecutadas por tests}}{\text{Total de líneas de código}} \times 100 \]
Una alta cobertura reduce la probabilidad de regresiones (bugs introducidos por nuevos cambios) y es un requisito común en los Quality Gates.

\section{OWASP Top 10}
El proyecto OWASP (Open Web Application Security Project) publica periódicamente una lista de los 10 riesgos de seguridad más críticos en aplicaciones web. En este laboratorio, nos centramos especialmente en:
\begin{itemize}
    \item \textbf{A03:2021 – Injection}: Ocurre cuando datos no confiables son enviados a un intérprete (como SQL) como parte de un comando o consulta. La remediación estándar es el uso de declaraciones preparadas (Prepared Statements).
    \item \textbf{A05:2021 – Security Misconfiguration}: Configuraciones de seguridad faltantes o incorrectas (ej. no usar cabeceras de seguridad o exponer trazas de error detalladas al usuario).
\end{itemize}