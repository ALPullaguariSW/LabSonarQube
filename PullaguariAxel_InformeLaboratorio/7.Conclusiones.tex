\chapter{Conclusiones}

La implementación de SonarQube en el ciclo de desarrollo de la "Vulnerable Parking App" transformó un código frágil e inseguro en un producto mantenible y confiable.

\begin{enumerate}
    \item Se demostró que la automatización del análisis de código es fundamental para detectar vulnerabilidades críticas como SQL Injection que son invisibles a simple vista.
    \item La refactorización guiada por métricas (Clean as You Code) no solo mejora la legibilidad, sino que reduce la superficie de ataque y facilita la evolución futura del software.
    \item El cumplimiento de un Quality Gate estricto (80\% de cobertura) obliga a adoptar prácticas de desarrollo disciplinadas, como TDD (Test Driven Development) o la escritura de tests post-implementación, lo cual eleva significativamente la confianza en el despliegue.
\end{enumerate}

En conclusión, herramientas como SonarQube son indispensables para cualquier equipo de ingeniería de software que aspire a estándares profesionales de calidad.
