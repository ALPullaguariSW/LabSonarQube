\chapter{Introducción}

La calidad del software no es un atributo accidental, sino el resultado de un proceso deliberado de diseño, codificación y verificación. En el desarrollo moderno, la deuda técnica —entendida como el costo implícito de retrabajar código debido a soluciones rápidas o malas prácticas— puede acumularse rápidamente, comprometiendo la seguridad, mantenibilidad y escalabilidad de los proyectos.

El análisis estático de código (SAST) se ha convertido en una práctica estándar para identificar estos problemas en etapas tempranas del ciclo de vida de desarrollo de software (SDLC). Herramientas como \textbf{SonarQube} permiten automatizar la revisión de código, detectando "code smells", vulnerabilidades de seguridad y errores lógicos que podrían pasar desapercibidos en revisiones manuales.

En el presente laboratorio, se aborda la auditoría y refactorización de una aplicación web heredada ("Vulnerable Parking App"), la cual presenta múltiples deficiencias de calidad. A través de la configuración de un entorno de análisis con Docker y SonarQube, se busca no solo identificar fallos críticos (como inyecciones SQL o exposiciones de datos sensibles), sino también implementar una cultura de "Clean Code" mediante la corrección sistemática de errores y la implementación de pruebas unitarias robustas para asegurar una cobertura de código aceptable.

El informe detalla el flujo de trabajo seguido: desde el despliegue de la infraestructura de análisis, pasando por la identificación y remediación de cientos de incidencias, hasta la consecución de un "Quality Gate" exitoso.
